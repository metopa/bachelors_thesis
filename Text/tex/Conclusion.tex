This thesis explores and extends the concept of the numerical database. First of all, an extensive overview of the original concept and its practical implementation is provided, basing on the \cite{park90}. In \Cref{ch:alt} the flaws of the original WST implementation are pointed out and the fix for them is presented.

In addition, the alternative data structures~-- the hash table and the splay tree, that can be used in place of WST, are discussed. The practical performance evaluation, presented in \Cref{ch:bench}, showed that the hash table achieves almost the same performance as WST. However, the splay tree is much slower than other data structures.

\Cref{ch:alt} introduces several alternative policies for managing item priorities~-- LRU and LFU policies. However, the benchmark showed that the original policy, based on the binary heap, is more effective.

This thesis presents new concurrent data structure~-- \cndcname.
It is based on the combination of the hash table and the binary heap. It has been adapted to the multi-threaded environment using the fine-grained locking approach (\Cref{sec:fgl}). Its advantage is the high scalability with increasing number of threads. However, it shows rather low single-thread performance ~-- this is due to the high memory overhead per node. Further researches, directed at lowering the overhead, should be done.

Finally, the \numdbname library is presented. It realises the data structures, mentioned above. The implementation is done in the C++ language. \numdbname has been used for the practical performance evaluation, presented in \Cref{ch:bench}. On the particular test, the following speed up has been achieved:

\pagebreak

\begin{itemize}
\item[] Base algorithm without numerical database~-- 588 operations per second
\item[] Splay tree based numerical database~-- 976 operations per second
\item[] Hash table based numerical database~-- 2407 operations per second
\item[] WST based numerical database~-- 2595 operations per second
\end{itemize}
